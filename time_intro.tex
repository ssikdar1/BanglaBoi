\documentclass{article}
\usepackage[banglamainfont=Kalpurush,
            banglattfont=Akaash
           ]{latexbangla}
\usepackage{multirow}
\title{ Quick Introduction to Expressing Time }
\author{Shantanu Sikdar}
\date{9/29/2019}

\begin{document}

\maketitle

%\section{Introduction}
%Here is the text of your introduction.

\subsection{Basic Vocbulary}
Here is some basic vocabulary. 

Today - আজ  (aaj) \\
Tomorrow - কাল (kaal) \\
Day after tomorrow - পরশু (porshoo) \\
Yesterday - গতকাল (gothokaal) but you can also use কাল (kaal) as well. \\
Usually the context and the tense is used to get the meaning of কাল. \\
For example: \\
কালকে আমি দোকানে গেলাম (kaalke ami dokan gelam) - Yesterday I went to the store \\
কালকে আমি দোকান যাবো (kaalke ami dokan jabo) - Tommorow I'll go to the store \\
\\
\\
Next - আগামী (aagaami) \\
Past - গত (gata) \\
Before - আগে (aagae) \\
\\
Month - মাস (maash) \\
1st of the month - \\
Last Month - গত মাসে (gotho maash) \\
Next Month - আগামী মাস (agamai maash) \\
Two months ago - দুই মাস আগে (dui mash agae) \\
\\
Year - বছর (bochhour) \\
Last Year - গত বছর (gotho bochhour) \\
Next Year - আগামী বছর (agami bochhour) \\
Two years ago - দুই বছর আগে (dui bochhour agae) \\
\\

\subsection{General Times of the Day}
The suffix বেলা is indicates a time period. It can be used with times of the day as defined below, but it can also be used in more general sense like ছেলেবেলা (chelebala) - childhood (ছেলে means boy or child ). \\

Daybreak - ভোর (bhor) \\
Early Morning Time - ভোর বেলা (bhor bala) \\
Morning - সকাল (shokal)
Morning time -  সকাল বেলা (shokal bala) \\
Noon - দুফুর (duphur) \\
Afternoon - বিকেল (bikel) \\
Afternoon time - বিকেলবেলা (bikel bala) \\
Evening - সন্ধ্যা (shondah)
Evening time - সন্ধ্যা বেলা (shondah bala) \\
Night - রাত (rath)
Night time - রাতের বেলা (raather bala)  \\

\subsection{Days of the Week}

\subsection{Months}

As far as the Georgian calendar goes the month names are just transliterations of the English names. \\


\begin{center}
\begin{tabular}{ c c c }
 January & জানুয়ারী  \\ 
 Febuary & ফেব্রুয়ারি  \\  
 March & মার্চ     \\
 April & এপ্রিল    \\
 May & মে    \\
 June & জুন    \\
 July & জুলাই   \\
 August & অগাস্ট  \\
 September & সেপ্টেম্বর  \\
 October & অক্টোবর  \\
 November & নভেম্বর  \\
 December & ডিসেম্বর  
\end{tabular}
\end{center}

However Bengal another set of months based on its own calendar: \\


\end{document}
